\documentclass[a4paper,11pt]{article}
\usepackage[top = 1.5 in, bottom = 1 in, left = 1 in, right = 1 in]{geometry}
\usepackage[utf8]{inputenc}
\usepackage[T1]{fontenc}
\usepackage{csquotes}
\usepackage[british]{babel}
\usepackage{lmodern}
\usepackage{url}
\usepackage{color}
\usepackage{graphicx}
\usepackage{setspace}
\usepackage{pdfpages}
\usepackage{amsmath}

\begin{document}
\title{Compiler project: MiniJava}
\author{Laura Leppänen \\ Compilers, Spring 2012}
\date{\today}
\maketitle
\thispagestyle{empty}

\tableofcontents
\onehalfspacing

\newpage
\setcounter{page}{1}

\section{Compiler implementation}

This section covers the general architecture of the compiler, testing, error handling as well as building and running instuctions.

\subsection{Architecture}

\subsection{Error handling}

\subsection{Testing}

\subsection{Building and running the compiler}

\section{The Mini-Java language}

\subsection{Token patterns}

Defined using regular expressions with Java style character classes. These token groups correspond to token classes used in the implementation.

\begin{description}
\item[Identifiers] $\backslash\text{p\{IsAlphabetic\}} [ \backslash\text{p\{IsAlphabetic\}}\backslash\text{p\{IsDigit\}\_ } ]^{*}$ \\ \emph{Except simple types and keywords.}
\item[Integer literals] $\backslash\text{p\{IsDigit\}}^{+}$
\item[Simple type] int | boolean | void
\item[Keyword] class | public | static | main | extends | assert | if | else | \\ while | System | out | println | return | new | length | this | true | false
\item[Operator] \&\& | || | < | > | == | + | - | *  | / | \% | = | !
\item[Punctuation] \{ | \} | [ | ] | ( | ) | $\backslash$. | ; | ,
\item[Single line comment] $\text{//.}^{*}\backslash \text{n}$ \\ \emph{Not a token but needs to be recognised by the scanner.}
\item[Multiline comment] $\text{/* .}^{*}\text{ */}$ \\ \emph{Not a token but needs to be recognised by the scanner and can be nested.}
\end{description}

\subsection{Modified grammar for recursive descent parsing (non-LL(1))}

In the grammar below, I have solved operator precedences using the ``classical'' method of making a separate production rule for each level of operator precedence. This is very easy to implement in the case of Mini-Java especially since all the defined binary operators are left-associative and there is only one unary operator. In a more complex case something like the Shunting Yard algorithm would probably work better.\footnote{Note: A program example on the course's grammar page also uses a unary minus operator, but this is not reflected in the grammar that was given, so I have left it out.}

A look-ahead of more than one token is needed e.g. when the parser sees an identifier in the input and is trying to parse a statement. In this case the result could be a local variable declaration for either a user defined type or an array of a user defined type or a statement that starts with an expression that begins with a variable reference.

\begin{verbatim}
<program>              ::= <main class> <class declaration list>
<main class>           ::= "class" <identifier> "{" "public" "static" "void" "main"
                           "(" ")" "{" <statement list> "}" "}"
<class declaration>    ::= "class" <identifier> <optional inheritance> "{"
                           <declaration list> "}"
<optional inheritance> ::= "extends" <identifier>
                        |  epsilon
<declaration>          ::= <variable declaration>
                        |  <method declaration>

<class declaration list> ::= <class declaration> <class declaration list>
                          |  epsilon
<declaration list>       ::= <declaration> <declaration list>
                          |  epsilon
<statement list>         ::= <statement> <statement list>
                          |  epsilon

<method declaration>   ::= "public" <type> <identifier> "(" <opt formals> ")"
                           "{" <statement list> "}"
<opt formals>          ::= <type> <identifier> <formals list>
                        |  epsilon
<formals list>         ::= "," <type> <identifier> <formals list>
                        |  epsilon
<variable declaration> ::= <type> <identifier> ";"
<type>                 ::= <simple type> <opt brackets>
<simple type>          ::= "int" | "boolean" | "void" | <type identifier>
<opt brackets>         ::= "[" "]" | epsilon
<type identifier>      ::= <identifier>

<statement>      ::= "assert" "(" <expr> ")" ";"
                  |  <local variable declaration>
                  |  "{" <statement list> "}"
                  |  "if" "(" <expr> ")" <statement> <opt else>
                  |  "while" "(" <expr> ")" <statement>
                  |  "System" "." "out" "." "println" "(" <expr> ")" ";"
                  |  "return" <expr> ";"
                  |  <expr> <opt assignment> ";"
<opt else>       ::= "else" <statement> | epsilon
<opt assignment> ::= "=" <expr> | epsilon
<local variable declaration> ::= <variable declaration>

<expr> ::= <or-operand> <or-operand-list>
<or-operand> ::= <and-operand> <and-operand-list>
<and-operand> ::= <eq-operand> <eq-operand-list>
<eq-operand>  ::= <neq-operand> <neq-operand-list>
<neq-operand> ::= <add-operand> <add-operand-list>
<add-operand> ::= <mult-operand> <mult-operand-list>
<mult-operand> ::= "!" <term>
                |  <term>

<or-operand-list> ::= "||" <or-operand> <or-operand-list>
                   |  epsilon
<and-operand-list> ::= "&&" <and-operand> <and-operand-list>
                    |  epsilon
<eq-operand-list> ::= "==" <eq-operand> <eq-operand-list>
                   |  epsilon
<neq-operand-list> ::= "<" <neq-operand> <neq-operand-list>
                    |  ">" <neq-operand> <neq-operand-list>
                    |  epsilon
<add-operand-list> ::= "+" <add-operand> <add-operand-list>
                    |  "-" <add-operand> <add-operand-list>
                    |  epsilon
<mult-operand-list> ::= "/" <mult-operand> <mult-operand-list>
                     |  "*" <mult-operand> <mult-operand-list>
                     |  "%" <mult-operand> <mult-operand-list>
                     |  epsilon

<term>    ::= "new" <new type> <opt term tail>
           |  "(" <expr> ")" <opt term tail>
           |  <identifier> <opt term tail>
           |  <integer literal> <opt term tail>
           |  "this" <opt term tail>
           |  "true" <opt term tail>
           |  "false" <opt term tail>

<new type>          ::= <simple type> "[" <expr> "]"
                     |  <type identifier> "(" ")"
<opt term tail>     ::= "[" <expr> "]" <opt term tail>
                     |  "." <method invocation> <opt term tail>
                     |  epsilon
<method invocation> ::= "length"
                     |  <identifier> "(" <opt exprs> ")"
<opt exprs>         ::= <expr list> | epsilon
<expr list>         ::= <expr> <expr list tail>
<expr list tail>    ::= "," <expr list> | epsilon

\end{verbatim}

\subsection{Abstract syntax trees}

\subsubsection{Interfaces}
\begin{description}
\item[ISyntaxTreeNode] \emph{is a node that can be visited.}
\item[IStatement] \emph{is an} ISyntaxTreeNode \emph{and represents a Mini-Java statement}
\item[IExpression] \emph{is an} ISyntaxTreeNode \emph{and represents a Mini-Java expression}
\end{description}

\subsubsection{Interface implementers}
\begin{description}
\item[Program] \emph{is an} ISyntaxTreeNode \\
\emph{is a root node.} \\
\emph{has a} \textbf{MainClassDeclaration} \\
\emph{has many} \textbf{ClassDeclaration}s
\\
\item[SyntaxElement] \emph{is an abstract} ISyntaxTreeNode \\
\emph{stores row and column information for nodes.}
\\
\item[MainClassDeclaration] \emph{is a} SyntaxElement \\
\emph{has a} \textbf{MethodDeclaration} \emph{which is the main method.} \\
\emph{defines a scope in the semantic analysis phase.}
\item[ClassDeclaration] \emph{is a} SyntaxElement \\
\emph{has many} \textbf{Declaration}s \\
\emph{defines a scope in the semantic analysis phase.}
\\
\item[Declaration] \emph{is an abstract} SyntaxElement \\
\emph{stores type information.}
\item[MethodDeclaration] \emph{is a} Declaration \\
\emph{has many} \textbf{VariableDeclaration}s \emph{which represent formal parameters.} \\
\emph{has many} \textbf{IStatement}s \emph{which form the method body.} \\
\emph{defines a scope in the semantic analysis phase.}
\item[VariableDeclaration] \emph{is a} Declaration \emph{and an} IStatement
\\
\item[AssertStatement] \emph{is a} SyntaxElement \emph{and an} IStatement \\
\emph{has an} \textbf{IExpression} \emph{which is the boolean argument.}
\item[BlockStatement] \emph{is a} SyntaxElement \emph{and an IStatement} \\
\emph{has many} \textbf{IStatement}s \emph{which form the body of the block.} \\
\emph{defines a scope in the semantic analysis phase.}
\item[IfStatement] \emph{is a} SyntaxElement \emph{and an} IStatement \\
\emph{has an} \textbf{IExpression} \emph{which represents the condition.} \\
\emph{has a} \textbf{BlockStatement} \emph{which is the then branch.} \\
\emph{has optionally a} \textbf{BlockStatement} \emph{which is the else branch.}
\item[WhileStatement] \emph{is a} SyntaxElement \emph{and an} IStatement \\
\emph{has an} \textbf{IExpression} \emph{which represents the condition.} \\
\emph{has a} \textbf{BlockStatement} \emph{which is the loop body}
\item[PrintStatement] \emph{is a} SyntaxElement \emph{and an} IStatement \\
\emph{has an} \textbf{IExpression} \emph{which is the integer argument.}
\item[ReturnStatement] \emph{is a} SyntaxElement \emph{and an} IStatement \\
\emph{has an} \textbf{IExpression} \emph{which is the expression to return.}
\item[MethodInvocation] \emph{is a} SyntaxElement \emph{and an} IStatement \emph{and an} IExpression \\
\emph{has an} \textbf{IExpression} \emph{which is the method owner} \\
\emph{has many} \textbf{IExpression}s \emph{which are the call parameters.}
\\
\item[ArrayIndexingExpression] \emph{is a} SyntaxElement \emph{and an} IExpression \\
\emph{has an} \textbf{IExpression} \emph{which is the array reference.} \\
\emph{has an} \textbf{IExpression} \emph{which is the index.}
\item[InstanceCreationExpression] \emph{is a} SyntaxElement \emph{and an} IExpression \\
\emph{has optionally an} \textbf{IExpression} \emph{which is the array size if this is an array creation.} \\
\emph{represents the 'new' expression.}
\item[ThisExpression] \emph{is a} SyntaxElement \emph{and an} IExpression \\
\emph{represents the reference to 'this' class.}
\item[VariableReferenceExpression] \emph{is a} SyntaxElement \emph{and an} IExpression
\item[UnaryOperatorExpression] \emph{is a} SyntaxElement \emph{and an} IExpression \\
\emph{has an} \textbf{IExpression} \emph{which is the operand.}
\item[BinaryOperatorExpression] \emph{is a} SyntaxElement \emph{and an} IExpression \\
\emph{has an} \textbf{IExpression} \emph{that is the left operand.} \\
\emph{has an} \textbf{IExpression} \emph{that is the right operand.}
\item[BooleanLiteralExpression] \emph{is a} SyntaxElement \emph{and an} IExpression
\item[IntegerLiteralExpression] \emph{is a} SyntaxElement \emph{and an} IExpression
\end{description}

\end{document}
